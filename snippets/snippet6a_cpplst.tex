\begin{lstlisting}[title=\href{https://godbolt.org/z/SGwrrW}{\texttt{godbolt.org/z/SGwrrW}}]
int f(int x) {
    return x + 1;
}

int g(int x) {
    return f(x + 2);
}
\end{lstlisting}
